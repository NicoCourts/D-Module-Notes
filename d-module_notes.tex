\documentclass[12pt]{article}

\usepackage{setspace}

\usepackage{amsmath, amsfonts, tikz-cd, mdframed, enumitem, framed, adjustbox, bbm, upgreek}
\usepackage[framed,thmmarks]{ntheorem}
\usepackage[style=alphabetic, bibencoding=utf8]{biblatex}
%Set the bibliography file
\bibliography{sources}


%Replacement for the old geometry package
\usepackage{fullpage}

%Input my definitions from this file.
%Theorems, lemmas, corollaries, and propositions
\theoremheaderfont{\normalfont\bfseries}
\theoremstyle{break}
\theorembodyfont{\itshape}
\theoremseparator{}
\theoremsymbol{}
\theoremindent0cm
\newtheorem{thm}{Theorem}[subsection]
\newtheorem{lem}[thm]{Lemma}
\newtheorem{cor}[thm]{Corollary}
\newtheorem{prop}[thm]{Proposition}

%Example
\theoremstyle{break}
\def\theoremframecommand{\colorbox[rgb]{0.9,0.9,0.9}}
\newshadedtheorem{ex}{Example}[section]

%proofs
\theoremstyle{nonumberbreak}
\theoremindent0cm
\theorempostskip{4ex}
\theoremheaderfont{\sc}
\theoremseparator{}
\theoremsymbol{\ensuremath\spadesuit}
\newtheorem{prf}{Proof}

\theoremstyle{nonumberplain}
\theorempostskip{0cm}
\theoremindent0cm
\theoremseparator{:}
\theoremsymbol{}
\newtheorem{conj}{Conjecture}

%remarks
\theoremstyle{change}
\theoremindent0cm
\theoremheaderfont{\sc}
\theoremseparator{:}
\theoremsymbol{}
\newtheorem{rmk}[thm]{Remark}

% Blackboard letters
\newcommand*{\bbA}{\mathbb{A}}
\newcommand*{\bbB}{\mathbb{B}}
\newcommand*{\bbC}{\mathbb{C}}
\newcommand*{\bbD}{\mathbb{D}}
\newcommand*{\bbE}{\mathbb{E}}
\newcommand*{\bbF}{\mathbb{F}}
\newcommand*{\bbG}{\mathbb{G}}
\newcommand*{\bbH}{\mathbb{H}}
\newcommand*{\bbI}{\mathbb{I}}
\newcommand*{\bbJ}{\mathbb{J}}
\newcommand*{\bbK}{\mathbb{K}}
\newcommand*{\bbL}{\mathbb{L}}
\newcommand*{\bbM}{\mathbb{M}}
\newcommand*{\bbN}{\mathbb{N}}
\newcommand*{\bbO}{\mathbb{O}}
\newcommand*{\bbP}{\mathbb{P}}
\newcommand*{\bbQ}{\mathbb{Q}}
\newcommand*{\bbR}{\mathbb{R}}
\newcommand*{\bbS}{\mathbb{S}}
\newcommand*{\bbT}{\mathbb{T}}
\newcommand*{\bbU}{\mathbb{U}}
\newcommand*{\bbV}{\mathbb{V}}
\newcommand*{\bbW}{\mathbb{W}}
\newcommand*{\bbX}{\mathbb{X}}
\newcommand*{\bbY}{\mathbb{Y}}
\newcommand*{\bbZ}{\mathbb{Z}}
%Fraktur letters
\newcommand*{\frakA}{\mathfrak{A}}
\newcommand*{\frakB}{\mathfrak{B}}
\newcommand*{\frakC}{\mathfrak{C}}
\newcommand*{\frakD}{\mathfrak{D}}
\newcommand*{\frakE}{\mathfrak{E}}
\newcommand*{\frakF}{\mathfrak{F}}
\newcommand*{\frakG}{\mathfrak{G}}
\newcommand*{\frakH}{\mathfrak{H}}
\newcommand*{\frakI}{\mathfrak{I}}
\newcommand*{\frakJ}{\mathfrak{J}}
\newcommand*{\frakK}{\mathfrak{K}}
\newcommand*{\frakL}{\mathfrak{L}}
\newcommand*{\frakM}{\mathfrak{M}}
\newcommand*{\frakN}{\mathfrak{N}}
\newcommand*{\frakO}{\mathfrak{O}}
\newcommand*{\frakP}{\mathfrak{P}}
\newcommand*{\frakQ}{\mathfrak{Q}}
\newcommand*{\frakR}{\mathfrak{R}}
\newcommand*{\frakS}{\mathfrak{S}}
\newcommand*{\frakT}{\mathfrak{T}}
\newcommand*{\frakU}{\mathfrak{U}}
\newcommand*{\frakV}{\mathfrak{V}}
\newcommand*{\frakW}{\mathfrak{W}}
\newcommand*{\frakX}{\mathfrak{X}}
\newcommand*{\frakY}{\mathfrak{Y}}
\newcommand*{\frakZ}{\mathfrak{Z}}
\newcommand*{\fraka}{\mathfrak{a}}
\newcommand*{\frakb}{\mathfrak{b}}
\newcommand*{\frakc}{\mathfrak{c}}
\newcommand*{\frakd}{\mathfrak{d}}
\newcommand*{\frake}{\mathfrak{e}}
\newcommand*{\frakf}{\mathfrak{f}}
\newcommand*{\frakg}{\mathfrak{g}}
\newcommand*{\frakh}{\mathfrak{h}}
\newcommand*{\fraki}{\mathfrak{i}}
\newcommand*{\frakj}{\mathfrak{j}}
\newcommand*{\frakk}{\mathfrak{k}}
\newcommand*{\frakl}{\mathfrak{l}}
\newcommand*{\frakm}{\mathfrak{m}}
\newcommand*{\frakn}{\mathfrak{n}}
\newcommand*{\frako}{\mathfrak{o}}
\newcommand*{\frakp}{\mathfrak{p}}
\newcommand*{\frakq}{\mathfrak{q}}
\newcommand*{\frakr}{\mathfrak{r}}
\newcommand*{\fraks}{\mathfrak{s}}
\newcommand*{\frakt}{\mathfrak{t}}
\newcommand*{\fraku}{\mathfrak{u}}
\newcommand*{\frakv}{\mathfrak{v}}
\newcommand*{\frakw}{\mathfrak{w}}
\newcommand*{\frakx}{\mathfrak{x}}
\newcommand*{\fraky}{\mathfrak{y}}
\newcommand*{\frakz}{\mathfrak{z}}
% Caligraphic letters
\newcommand*{\calA}{\mathcal{A}}
\newcommand*{\calB}{\mathcal{B}}
\newcommand*{\calC}{\mathcal{C}}
\newcommand*{\calD}{\mathcal{D}}
\newcommand*{\calE}{\mathcal{E}}
\newcommand*{\calF}{\mathcal{F}}
\newcommand*{\calG}{\mathcal{G}}
\newcommand*{\calH}{\mathcal{H}}
\newcommand*{\calI}{\mathcal{I}}
\newcommand*{\calJ}{\mathcal{J}}
\newcommand*{\calK}{\mathcal{K}}
\newcommand*{\calL}{\mathcal{L}}
\newcommand*{\calM}{\mathcal{M}}
\newcommand*{\calN}{\mathcal{N}}
\newcommand*{\calO}{\mathcal{O}}
\newcommand*{\calP}{\mathcal{P}}
\newcommand*{\calQ}{\mathcal{Q}}
\newcommand*{\calR}{\mathcal{R}}
\newcommand*{\calS}{\mathcal{S}}
\newcommand*{\calT}{\mathcal{T}}
\newcommand*{\calU}{\mathcal{U}}
\newcommand*{\calV}{\mathcal{V}}
\newcommand*{\calW}{\mathcal{W}}
\newcommand*{\calX}{\mathcal{X}}
\newcommand*{\calY}{\mathcal{Y}}
\newcommand*{\calZ}{\mathcal{Z}}
% Script Letters
\newcommand*{\scrA}{\mathscr{A}}
\newcommand*{\scrB}{\mathscr{B}}
\newcommand*{\scrC}{\mathscr{C}}
\newcommand*{\scrD}{\mathscr{D}}
\newcommand*{\scrE}{\mathscr{E}}
\newcommand*{\scrF}{\mathscr{F}}
\newcommand*{\scrG}{\mathscr{G}}
\newcommand*{\scrH}{\mathscr{H}}
\newcommand*{\scrI}{\mathscr{I}}
\newcommand*{\scrJ}{\mathscr{J}}
\newcommand*{\scrK}{\mathscr{K}}
\newcommand*{\scrL}{\mathscr{L}}
\newcommand*{\scrM}{\mathscr{M}}
\newcommand*{\scrN}{\mathscr{N}}
\newcommand*{\scrO}{\mathscr{O}}
\newcommand*{\scrP}{\mathscr{P}}
\newcommand*{\scrQ}{\mathscr{Q}}
\newcommand*{\scrR}{\mathscr{R}}
\newcommand*{\scrS}{\mathscr{S}}
\newcommand*{\scrT}{\mathscr{T}}
\newcommand*{\scrU}{\mathscr{U}}
\newcommand*{\scrV}{\mathscr{V}}
\newcommand*{\scrW}{\mathscr{W}}
\newcommand*{\scrX}{\mathscr{X}}
\newcommand*{\scrY}{\mathscr{Y}}
\newcommand*{\scrZ}{\mathscr{Z}}


%General purpose stuff
\DeclareMathOperator{\Aut}{Aut}
\DeclareMathOperator{\ch}{char}
%\let\char\relax
%\DeclareMathOperator{\char}{char}
\DeclareMathOperator{\rank}{rank}
\DeclareMathOperator{\End}{End}
\let\Im\relax
\DeclareMathOperator{\Im}{Im}
\newcommand{\eqdef}{\stackrel{\text{\tiny def}}{=}}
\DeclareMathOperator{\tr}{tr}
\DeclareMathOperator{\diag}{diag}
\DeclareMathOperator{\Gal}{Gal}

%Category Theory
\DeclareMathOperator{\Hom}{Hom}
\DeclareMathOperator{\iHom}{\mathscr {H}\kern -2pt om}
\DeclareMathOperator{\uHom}{\underline \Hom}
\DeclareMathOperator{\Mor}{Mor}
\let\hom\relax
\DeclareMathOperator{\hom}{hom}
\DeclareMathOperator{\id}{id}
\DeclareMathOperator{\coker}{coker}
\DeclareMathOperator*{\colim}{colim}
\DeclareMathOperator{\invlim}{\lim_{\leftarrow}}
\DeclareMathOperator{\dirlim}{\lim_{\rightarrow}}
\newcommand*{\ladjointto}{\mathrel{{\scriptstyle\dashv}}}
\newcommand*{\radjointto}{\mathrel{{\scriptstyle\vdash}}}

%Groups/Group Schemes
\newcommand{\Ga}{\bbG_a}
\newcommand{\Gm}{\bbG_m}
\DeclareMathOperator{\GL}{GL}
\DeclareMathOperator{\Sym}{Sym}
\newcommand{\red}[1]{{#1}_{\text{red.}}}

%Commutative Algebra\.
\DeclareMathOperator{\gldim}{gldim}
\DeclareMathOperator{\projdim}{projdim}
\DeclareMathOperator{\injdim}{injdim}
\DeclareMathOperator{\findim}{findim}
\DeclareMathOperator{\flatdim}{flatdim}
\DeclareMathOperator{\depth}{depth}

%Common Categories
\newcommand{\op}{^{\text{op}}}
\newcommand{\rmod}[1]{\mathbf{mod}\text{-}#1}
\newcommand{\lmod}[1]{#1\text{-}\mathbf{mod}}
\newcommand{\rMod}[1]{\mathbf{Mod}\text{-}#1}
\newcommand{\lMod}[1]{#1\text{-}\mathbf{Mod}}
\newcommand{\rcomod}[1]{\mathbf{comod}\text{-}#1}
\newcommand{\lcomod}[1]{#1\text{-}\mathbf{comod}}
\DeclareMathOperator{\Vectk}{\Vect_k}
\DeclareMathOperator{\Ch}{\mathbf{Ch}}
\newcommand*{\Chb}{\mathbf{Ch}^\text{b}}
\newcommand*{\Ab}{\mathbf{Ab}}
\newcommand*{\Grp}{\mathbf{Grp}}
\newcommand*{\Algk}{\mathbf{Alg}_k}
\newcommand*{\Alg}{\ensuremath{\mathbf{Alg}}}
\newcommand*{\Ring}{\mathbf{Ring}}
\newcommand*{\K}{\mathbf{K}}
\newcommand*{\D}{\mathbf{D}}
\newcommand*{\Db}{\mathbf{D}^\text{b}}
\newcommand*{\Dpos}{\mathbf{D}^+}
\newcommand*{\Dneg}{\mathbf{D}^-}
\newcommand*{\Dbperf}{\mathbf{D}^b_{\text{perf}}}
\newcommand*{\Dperf}{\mathbf{D}_{\text{perf}}}
\newcommand*{\Dsing}{\mathbf{D}_{sing}}
\newcommand{\CRing}{\mathbf{CRing}}
\DeclareMathOperator{\stmod}{\mathbf{stmod}}
\DeclareMathOperator{\StMod}{\mathbf{StMod}}
\DeclareMathOperator{\sHom}{\underline{Hom}}
\newcommand*{\Set}{\mathbf{Set}}
\newcommand*{\1}{\mathbbm{1}}
\newcommand{\Sch}{\mathbf{Sch}}
\newcommand{\Aff}{\mathbf{Aff}}
\newcommand{\GrpSch}{\mathbf{GrpSch}}
\newcommand{\AlgGrp}{\mathbf{AlgGrp}}
\newcommand{\Rep}{\mathbf{Rep}\,}
\newcommand{\Open}{\mathbf{Open}}
\newcommand{\PreSh}{\mathbf{PreSh}}
\newcommand{\Sh}{\mathbf{Sh}}
\newcommand{\Func}{\mathbf{Func}}
\newcommand{\Vect}{\mathbf{Vect}}


%Homological algebra
\DeclareMathOperator{\cone}{cone}
\DeclareMathOperator{\HH}{HH}
\DeclareMathOperator{\Der}{Der}
\DeclareMathOperator{\Ext}{Ext}
\DeclareMathOperator{\Tor}{Tor}
\newcommand{\R}{\mathbf{R}}
\let\L\relax
\newcommand{\L}{\mathbf{L}}

%Representation theory
\DeclareMathOperator{\Ind}{Ind}
\DeclareMathOperator{\Res}{Res}

%Algebraic Geometry
\DeclareMathOperator{\Spec}{Spec}
\DeclareMathOperator{\Proj}{Proj}
\let\O\relax
\newcommand*{\O}{\calO}
\newcommand{\A}[1]{\bbA^{#1}}
\let\P\relax
\newcommand{\P}[1]{\bbP^{#1}}
\DeclareMathOperator{\supp}{supp}

%Lie algebras
\DeclareMathOperator{\ad}{ad}
\newcommand*{\gl}{\mathfrak{gl}}
\let\sl\relax
\newcommand*{\sl}{\mathfrak{sl}}
\let\sp\relax
\newcommand*{\sp}{\mathfrak{sp}}
\newcommand*{\so}{\mathfrak{so}}
\DeclareMathOperator{\Lie}{Lie}

% Hacks and Tweaks
% Enumerate will automatically use letters (e.g. part a,b,c,...)
%\setenumerate[0]{label=(\alph*)}
% Always use wide tildes
\let\tilde\relax
\newcommand*{\tilde}[1]{\widetilde{#1}}
%raise that Chi!
\DeclareRobustCommand{\Chi}{{\mathpalette\irchi\relax}}
\newcommand{\irchi}[2]{\raisebox{\depth}{$#1\chi$}} 



%%%%%%%%%%%%%%%%%%%%%%%%%%%%%%%%%%%%%%%%%%%%%%%%%%%%%%%%%%%%%%%%%%%%%%
%%%%%%%%%%%%%%%%%%%%%%% Customize Below %%%%%%%%%%%%%%%%%%%%%%%%%%%%%%
%%%%%%%%%%%%%%%%%%%%%%%%%%%%%%%%%%%%%%%%%%%%%%%%%%%%%%%%%%%%%%%%%%%%%%

% Document-Specific Macros
% Feel free to add to this.
\newcommand{\Bbbk}{\mathbbm{k}}
\let\AA\relax
\newcommand{\AA}{\bbA}
\newcommand{\CC}{\bbC}
\newcommand{\PP}{\bbP}
\newcommand{\cD}{\mathcal D}
\newcommand{\cO}{\mathcal O}


\begin{document}
%make the title page
\title{D-modules}
\author{A seminar at University of Washington}
\date{Winter 2020}
\maketitle

\section{The basics}

We work over an algebraically closed field $\Bbbk$, always of characteristic zero, usually 
$\bbC$.\footnote{The positive characteristic theory of differential operators  
is completely different from that in characteristic zero. For example, when ${\rm char}(\Bbbk)>0$, the ring of differential operators on
the polynomial ring $\Bbbk[x]$ is not finitely generated, not noetherian, and not a domain.}

The letter $X$ will denote either a smooth quasi-projective variety, or a non-singular analytic variety, or a complex manifold, and 
$\calO_X$ will denote the sheaf of regular functions, or analytic/holomorphic functions, or smooth functions.


\subsection{The first Weyl algebra}
Consider  the case $X=\bbA^n$, affine $n$-space. The ring of regular functions on it is the polynomial ring $\Bbbk[x_1,\ldots,x_n]$. 
We view $\Bbbk[x_1,\ldots,x_n]$ as acting on itself by multiplication. In that way $\Bbbk[x_1,\ldots,x_n]$ is identified with a subalgebra
of $\End_\Bbbk(\Bbbk[x_1,\ldots,x_n])$, the ring of $\Bbbk$-linear transformations from $\Bbbk[x_1,\ldots,x_n]$ to itself. 
The usual partial derivative operators $\partial_i=\frac{\partial}{\partial x_i}$ belong to $\End_\Bbbk(\Bbbk[x_1,\ldots,x_n])$. The {\sf
ring of differential operators} on $\AA^n$, or $\Bbbk[x_1,\ldots,x_n]$, is defined to be the subalgebra of $\End_\Bbbk(\Bbbk[x_1,\ldots,x_n])$
generated by the $\partial_i$'s and the aforementioned copy of $\Bbbk[x_1,\ldots,x_n]$. We denote it by
$$
\calD(\bbA^n) \; :=\; \Bbbk[x_1,\ldots,x_n,\partial_1,\ldots,\partial_n].
$$
This ring is commonly called the {\sf $n^{\rm th}$ Weyl algebra}, presumably in honor of Hermann Weyl because it appears in his 1931 
book {\it Gruppentheorie und Quantenmechanik}. 

When $n=1$, $\calD(\bbA^n)$ is called the {\sf first Weyl algebra}. It was one of the first infinite-dimensional non-commutative 
algebras to be examined
in any detail \cite[Part II]{littlewood33}. There it was viewed as an algebra presented by generators and relations: 
the algebra $\bbC[x,\partial]$ (we write $\partial$ rather than $\partial_x$ when $n=1$) is isomorphic to 
$$
\frac{\bbC\langle x,p \rangle}{(px-xp-1)}\, ,
$$
i.e., the quotient of the free algebra modulo the relation $px-xp=1$. The letter $p$ was chosen because at that time the letter $P$ was
used to denote the momentum operator in Heisenberg's formulation of quantum mechanics.\footnote{Did Heisenberg choose the letter 
$p$ because of the word ``partial''?} Heisenberg writes $\hat{x}$
for the position operator and $\hat{p}$ for the momentum operator $-i\hbar\frac{\partial}{\partial x}$. The fact that 
$[\hat{x},\hat{p}]=i\hbar$ as interpreted by Heisenberg as the ``uncertainty principle''.
In \cite[Part II]{littlewood33} the author takes note of Heisenberg's 1925 matrix interpretation
\begin{equation}
\label{matrix.opers}
x \;=\; \begin{pmatrix}
  0    &&    \\
   1   &&  \\
   & 1 && \\
   && 1 & \\
   &&& \ddots
\end{pmatrix}
\qquad 
p \; :=\; \begin{pmatrix}
  0  & 1    \\
      & & 2 \\
      &&& 3 \\  
      &&&& \ddots
\end{pmatrix}
\end{equation}
that plays a foundational role in Heisenberg's  formulation of matrix mechanics.

In modern terms, the natural left action of $\calD(\bbA^1)$ on $\bbC[x]$ makes the latter a left module over the former and, with 
respect to the ordered basis $\{1,x,x^2,\ldots,\}$, the elements $x$ and $\partial$ act via the matrices $x$ and $p$ respectively. Check that 
the matrices satisfy the relation $px-xp=1$. 

Notice that the relation $\partial x - x \partial =1$, where $1$ denotes the operator ``multiplication by 1'' on $\bbC[x]$, is a 
consequence of the product rule for differentiation, viz., $(\partial x -x \partial)(f)=(xf)'-x f'=f +xf'-xf'=f$.


\subsubsection{Exercises}
Sometimes we will write $A$ for $\bbC[x,\partial]$ to shorten the notation.

\begin{enumerate}
  \item 
  Show that  $\{x^i\partial^j \; | \; i,j \ge 0\}$ is a basis for  the subalgebra of $\End(\bbC[x])$ generated by $x$, viewed as the operator ``multiplication by $x$'', and $\partial$, i.e., as a basis for $\bbC[x,\partial]$.
  
  
  As a consequence of this one sees that $\bbC[x,\partial]$ contains copies of the polynomial rings $\bbC[x]$ and $\bbC[\partial]$.
  Moreover, $\bbC[x,\partial]$ is a free left $\bbC[x]$-module with basis $1,\partial,\partial^2,\ldots$ and 
  a free right $\bbC[\partial]$-module with basis $1,x,x^2,\ldots$.
  
  Show that $\bbC[x,\partial]$ is a free {\it right} $\bbC[x]$-module with basis $1,\partial,\partial^2,\ldots$ and 
  a free {\it left} $\bbC[\partial]$-module with basis $1,x,x^2,\ldots$.
  \item 
 Show that the kernel of the homomorphism  $\bbC\langle x,p \rangle \to \bbC[x,\partial]$ given by $x \mapsto x$ and $p \mapsto \partial$
 is the ideal generated by $px-xp-1$.
  \item 
In  $\bbC[x,\partial]$, show that $[\partial,f]=f'$ for all $f \in \bbC[x] \subseteq  \bbC[x,\partial]$.
\item
Show that $\bbC[x,\partial]$ is a free left $\bbC[x\partial]$-module with basis $\ldots,\partial^2,\partial,1,x,x^2,\ldots$.
\item
\label{ex.grading}
Show that $\bbC[x,\partial]$ is a $\bbZ$-graded algebra with $\deg(\partial)=-1$ and $\deg(x)=+1$ and determine the homogeneous
components with respect to this grading.\footnote{With this grading, the category of graded left $\bbC[x,\partial]$-modules is
equivalent to the category of quasi-coherent sheaves on a certain stack.}
\item
For $n \ge 0$ define 
$$
F_n \; :=\; \text{the left $\bbC[x]$-submodule of $\bbC[x,\partial]$ generated by $1,\partial,\ldots,\partial^n$}.
$$
Observe that $F_0 \subseteq F_1\subseteq \ldots \subseteq F_n$ and $\cup_{n=0}^\infty F_n = \bbC[x,\partial]$.
The $F_n$'s therefore give $\bbC[x,\partial]$ the structure of a {\it filtered ring}. We say that the $F_n$'s provide 
a {\sf filtration} on $\bbC[x,\partial]$.

The {\sf order} of an element $a \in \bbC[x,\partial]$ is the smallest $n$ such that $a \in F_n$. It is often convenient to define $F_n=\{0\}$ for
all $n<0$ and define the order of 0 to be $-\infty$.

Show that ${\rm order}(ab)={\rm order}(a) + {\rm order}(b)$ and hence that $\bbC[x,\partial]$ is a domain (i.e. a product in
$\bbC[x,\partial]$ is 0 if and only if one of the factors is 0).
\begin{enumerate}
  \item
  Show that $F_iF_j=F_{i+j}$ for all $i,j \ge 0$.  
  \item 
  Show that $[a,b] \in F_{i+j-1}$ if $a \in F_i$ and $b \in F_j$.
  \item 
  Define $G_n:=F_n/F_{n-1}$ and $G:= \oplus_{n \ge 0} G_n$. Show that $G$ becomes a ring under the product
  $$
  (a+F_{m-1})(b+F_{n-1}) \; :=\; ab+F_{m+n-1},
  $$
  i.e., check this multiplication is well-defined and associative.
  
The ring $G$ is called {\sf the associated graded ring} of $\bbC[x,\partial]$ and denoted ${\sf gr}\, \bbC[x,\partial]$.
  
  If $a \in \bbC[x,\partial]$ has order $n$ we call the element $a+F_{n-1}$ the {\sf symbol} of $a$.
  \item
  Show that ${\sf gr}\, \bbC[x,\partial]$ is commutative.
  \item
   Show that ${\sf gr} \, \bbC[x,\partial]$ is a polynomial ring on two variables, i.e., that it is generated by the ``indeterminates''
   $x+F_{-1}$ and $\partial+F_0$. 
\end{enumerate}
\item
Let $F_0 \subseteq F_1 \subseteq \cdots$ be a filtration on a ring $R$. Show that $R$ is a domain if ${\sf gr} R$ is.
\item
Let $F_0 \subseteq F_1 \subseteq \cdots$ be a filtration on a ring $R$. Show that $R$ is left noetherian if ${\sf gr} R$ is.

As a consequence of this, $\bbC[x,\partial]$ is a left and right noetherian ring.
\item
Show that $\bbC[x,\partial]$ is a simple ring, i.e., that its only two-sided ideals are itself and $\{0\}$. Hint: let $I$ be a non-zero ideal\footnote{The word {\it ideal} without the qualifier {\it left} or {\it right} always means {\it two-sided ideal}. } and choose a
non-zero element $a \in I$ of minimal order; if that order is $>0$ show that $[x,a] \ne 0$ and ${\rm order}([x,a])<{\rm order}(a)$
and therefore deduce that ${\rm order}(a)=0$; 
now choose a non-zero $a \in \bbC[x] \cap I$ of minimal degree and show that if $\deg(a)>0$, then $[\partial,a]\ne 0$ and
$\deg([\partial,a])<\deg(a)$; conclude that $\bbC[x] \cap I$ contains a non-zero element of degree 0; such elements belong to $\CC-\{0\}$ 
so conclude that $I=\bbC[x,\partial]$.
\item
Show that the only finite dimensional left $\bbC[x,\partial]$-module is $\{0\}$. 
\item
\label{ex.simple.module}
Let $A=\bbC[x,\partial]$.

By definition, $\bbC[x,\partial]$ is a subalgebra of $\End_\bbC(\bbC[x])$ so $\bbC[x]$ is a left $\bbC[x,\partial]$-module
with $x$ acting on it by multiplication and $\partial$ acting on as $d/dx$. Show that $\bbC[x]$ is a simple (or, synonymously, an
irreducible) module; i.e., that its only submodules are itself and $\{0\}$. Since $1 \in \bbC[x]$ is a generator for that module, 
$\bbC[x] \cong A/I$ where
$$
I \;=\; \{a \in A \; | \; a \cdot 1=0\}.
$$
(This is general ring theory, as is the fact that $I$ is a left ideal, in fact a maximal left ideal because $\bbC[x]$ is a simple module.)
Show that $I=A\partial$. 

Notice that $\bbC[x]$ becomes a graded left $\bbC[x,\partial]$-module when $\bbC[x,\partial]$ is given the grading in 
Exercise \ref{ex.grading} and $\bbC[x]$ is given the grading $\deg(x^n):=n$. It might be more satisfying to say that 
$\bbC[x,\partial]$ inherits its grading from the standard grading on $\bbC[x]$ by declaring that $a \in \bbC[x,\partial]$ is
homogeneous of degree $n$ if $a\cdot x^i \in \bbC x^{i-n}$ for all $i \in \bbN$. 

\item
Show that $\bbC[x,\partial]$ is anti-isomorphic to itself, i.e., that it is isomorphic to its opposite $\bbC[x,\partial]^{\circ}$, via the map
$x \mapsto -\partial$ and $\partial \mapsto x$. This anti-isomorphism interchanges the subalgebras $\bbC[x]$
and $\bbC[\partial]$ so can be thought of as an algebraic analogue of the Fourier transform.
\end{enumerate}

\bigskip


The exercises establish some of the basic properties $\bbC[x,\partial]$: it is left and right noetherian, a domain, and a simple ring.
Another  of its important properties is that it has global homological dimension one, meaning that every left
$\bbC[x,\partial]$-module $M$ has a projective resolution of the form $0 \to P_1 \to P_0 \to M \to 0$. 
This is in sharp contrast to the fact that ${\sf gr}\, \bbC[x,\partial]$ has global dimension 2. There are other properties 
that $\bbC[x,\partial]$ shares with Dedekind domains. 
Every left ideal in $\bbC[x,\partial]$ is projective. Every left ideal in $\bbC[x,\partial]$ is 
generated by $1\frac{1}{2}$ elements: if $I$ is a non-zero left ideal and $a$ is any non-zero element in it, 
then $I$ is generated as a left ideal by 
$a$ and one more element. If $P_1$ and $P_2$ are finitely generated projective left $\bbC[x,\partial]$-modules, then 
$P_1 \oplus P_2$ is a free module. Every  finitely generated projective left $\bbC[x,\partial]$-module is isomorphic to the direct
sum of a left ideal and a free module. Every finitely generated left $\bbC[x,\partial]$-module is a direct sum of a projective
module and a torsion module (a left $\bbC[x,\partial]$-module $M$ is {\sf torsion} if for each $m \in M$ there is a non-zero 
$a \in  \bbC[x,\partial]$ such that $am=0$).

The question of whether every left ideal of $\bbC[x,\partial]$ is generated by a single element was considered in the 1930's.\footnote{I don't recall when or by whom. \"Ore? Does anyone know a reference.} The first non-principal left ideal  that was found is  
$I=A\partial^2+A(x\partial -1)$.  (Here $A=\bbC[x,\partial]$.) We note that $I$ is a maximal left ideal because $\bbC[x]$ is a simple
$A$-module (Exercise \ref{ex.simple.module}) and $I$ is the annihilator of $x$ which {\it must} generate the simple module 
$\bbC[x]$. Because the generators of $I$ are homogeneous with respect to the grading in Exercise \ref{ex.grading} one can exploit
that grading to show that $I$ is not a principal left ideal.\footnote{I do not know if that is the original proof.}\footnote{There is an 
isomorphism of left $A$-modules $A/I \cong A/A\partial$ even though $I \ne A\partial$; indeed, $I$ is not even isomorphic to
$A\partial$. This sort of behavior is normal in non-commutative rings.}




\subsubsection{Solutions to differential equations}
When solving ordinary differential equations one rarely confines one's attention to polynomial solutions.
More typically one might ask for solutions that are holomorphic or meromorphic. 
It is therefore natural to replace $\bbC[x]$ by the $\bbC$-algebra 
of holomorphic or meromorphic functions and to replace $\bbC[x,\partial]$ by the 
larger algebra generated by that ring of functions and the operator $\partial$.

Let $K$ denote the field of meromorphic functions on $\bbC$ (we replace $x$ by $z$) and write 
$\partial$ for $d/dz$. We replace $\bbC[z,\partial]$ by $K[\partial]$ which can be defined as the subalgebra of $\End_\bbC(K)$
generated by $K$, acting on itself by multiplication, and $\partial$. The natural action of $K[\partial]$ makes $K$ a left
$K[\partial]$-module.



Consider an ordinary differential equation 
\begin{equation}
\label{diffl.eq.1}
a_n(z)\frac{d^n f}{dz^n} + \cdots + a_1(z)\frac{d f}{dz} +a_0(z)f \;=\; 0.
\end{equation}
where the $a_i(z)$'s are meromorphic functions. The differential operator 
$$
D \; :=\; a_n(z)\frac{d^n }{dz^n} + \cdots + a_1(z)\frac{d }{dz} +a_0(z)
$$
belongs to $K[\partial]$ and the solutions to (\ref{diffl.eq.1})  are the elements in the module $K$ that are annihilated
by $D$, i.e., those $f$ such that $D \cdot f=0$.


\subsubsection{Monodromy}
Typically, we do not expect to have solutions that are meromorphic on the whole complex plane. For that reason we might replace 
$K$ by functions that are holomorphic or meromorphic on some domain, $U$ say, (i.e., a connected open subset of $\CC$).  
It is therefore natural to replace $K$ by $\cO(U)$, the ring of holomorphic functions on $U$. Questions of analytic continuation and
monodromy arise at this point. 

The simplest example illustrating this is the differential equation
$$
\frac{f'(z)}{f(z)} \; = \; \frac{1}{2z}
$$
on the domain $\CC^\times = \CC-\{0\}$. In a small open ball around $1$, $f(z)=\sqrt{z}$ is a solution. 


{\color{red} say more about this}



\subsubsection{The first Weyl algebra over a field of positive characteristic}


\subsection{The higher Weyl algebras}



\subsection{The ring of differential operators on a non-singular affine variety} 
Fix a non-singular affine algebraic variety $X$ over an algebraically closed field $\Bbbk$ of characteristic zero.
When we say $X$ is a {\it variety} we mean it is a reduced and irreducible scheme; i.e., its coordinate ring, which we denote by 
$\calO(X)$, has no non-zero zero-divisors. 

A {\sf $\Bbbk$-linear derivation} on $\calO(X)$ is a $\Bbbk$-linear map $\delta:\calO(X) \to \calO(X)$ such that
$$
\delta(ab) \;=\; \delta(a)b+a\delta(b)
$$
for all $a,b \in \calO(X)$. The set of all derivations is denoted by ${\rm Der}\, \calO(X)$ and it is made into a left $\calO(X)$-module by
declaring that $f\delta$ is the map $a \mapsto f\, \delta(a)$.

It is easy to see that $[\delta_1,\delta_2]$ is a derivation if $\delta_1$ and $\delta_2$ are. 
Thus, ${\rm Der}\, \calO(X)$ is a Lie algebra.

Derivations are the algebraist's vector fields. The statement that the sheaf (ring) of differential operators on a smooth (affine) variety
is generated by the sheaf (ring) of functions on it and vector fields is equivalent to the statement that it is generated by functions and 
vector fields. Derivations make sense for any ring so such an algebraic translation is necessary in order to transfer
differential geometric ideas and tools to settings like positive characteristic geometry, $p$-adic algebras, and so on.



We define
$$
\calD(X) \;:=\; \text{the subalgebra of $\End_\Bbbk(\calO(X))$ generated by $\calO(X)$ and ${\rm Der}\, \calO(X)$}.
$$
In this definition we view $\calO(X)$ as a subalgebra of $\End_\Bbbk(\calO(X))$ by having it act on itself by multiplication.

\begin{prop}
With the assumptions on $X$ above, $\calD(X)$ has the following properties:
\begin{enumerate}
  \item 
  it is a finitely generated $\Bbbk$-algebra;
  \item 
  it is left and right noetherian;
  \item 
  it is a domain;
  \item
  it is a simple ring;
  \item
  its global dimension is ${\rm dim}(X)$.
\end{enumerate}
\end{prop}

Several remarks are in order. 
\begin{enumerate}
  \item 
  When $X=\AA^n$, $\calD(\AA^n)$ is the $n^{\rm th}$ Weyl algebra $\Bbbk[x_1,\ldots,x_n,\partial_1,\ldots,\partial_n]$ 
  because 
  $$
  {\rm Der}(\Bbbk[x_1,\ldots,x_n]) \;=\; \bigoplus_{i=1}^n \Bbbk[x_1,\ldots,x_n] \, \frac{\partial}{\partial x_i} \, .
  $$
  \item 
  The fact that $\calD(X)$ is a finitely generated $\Bbbk$-algebra is a consequence of the fact that ${\rm Der} \calO(X)$
  is a finitely generated left $\calO(X)$-module. The best way to prove this is to start with the fact that the module of K\"ahler
  differentials $\Omega_{X/\Bbbk}$ is finitely generated and then proving that ${\rm Der} \calO(X) \cong
  \Hom_{\calO(X)}(\Omega_{X/\Bbbk}$. We fill in the details in \S\ref{ssect.Kahler.diffls} below.
  \item 
\end{enumerate}


\subsubsection{The module of K\"ahler differentials and the module of derivations}
\label{ssect.Kahler.diffls} 


\subsubsection{Differential operators behave well under localization}





\subsubsection{Grothendieck's definition of the ring of differential operators}
Fix a commutative ring $\Bbbk$ and a commutative $\Bbbk$-algebra $A$. Let $M$ and $N$ be $A$-modules. We give
$\Hom_\Bbbk(M,N)$ the structure of a left $A \otimes_\Bbbk A$-module by declaring that 
$$
((a \otimes b) \cdot \theta))(m) \; :=\; a\theta(bm)
$$
for $a,b \in A$ and $\theta \in \Hom_\Bbbk(M,N)$.
Let $J$ be the kernel of the multiplication homomorphism $\mu:A \otimes_\Bbbk A \to A$, $\mu(a \otimes b):=ab$. 


For $n \ge -1$, the space of {\sf $\Bbbk$-linear differential operators from $M$ to $N$ of order $\le n$} is 
$$
\cD^n(M,N) \; :=\; \{ \theta \in \Hom_\Bbbk(M,N) \; | \; J^{n+1}\theta =0\}.
$$
The space of {\sf $\Bbbk$-linear differential operators from $M$ to $N$} is
$$
 \cD_A(M,N) \; :=\; \bigcup_{n=-1}^\infty \cD^n(M,N).
 $$
It is clear that
$$
\{0\} \, = \, \cD^{-1}(M,N) \subseteq \cD^0(M,N) \subseteq  \cD^1(M,N) \subseteq \cdots \subseteq \cD^n(M,N) \subseteq \cdots 
$$
Since $J$ is generated by $\{a \otimes 1-1 \otimes a\}$, $\theta$ belongs to $\cD^0(M,N)$ if and only if it is an $A$-module
homomorphism. Thus, $\cD^0(M,N)=\Hom_A(M,N)$.\footnote{It is straightforward to check that $ \cD_A(M,N)$ is an $\End_\Bbbk(N)$-$\End_\Bbbk(M)$ sub-bimodule of $\Hom_\Bbbk(M,N)$.
(Recall that $\Hom_\Bbbk(M,N)$ is an $\End_\Bbbk(N)$-$\End_\Bbbk(M)$ bimodule with the action given by composition of functions.)


To check one's understanding one should show that a $\Bbbk$-linear map $\theta:M \to N$ belongs to $\cD^n(M,N)$ if and only if 
$[a,\theta] \in \cD^{n-1}(M,N) $ for all $a \in A$. This fact allows one to give a short inductive definition of $\cD_A(M,N)$ that does not involve
mentioning $A \otimes_\Bbbk A$, $J$, and so on.}

When $M=N$ we write $\cD_A(M)$ for $\cD_A(M,M)$. It is a subalgebra of $\End_\Bbbk(M)$. Notice that $ \cD_A(M,N)$ is a $\cD(N)$-$\cD(M)$-bimodule 

When $M=N=A$ we  write $\cD(A)$ for $\cD_A(A,A)$ and call it {\sf the ring of differential operators on $A$}.
When $X$ is a quasi-projective variety  we will write $\cD(X)$ for $\cD(\cO(X))$.


It is a nice exercise to show that $\cD^1(A,A) = A \oplus \Der_\Bbbk(A)$. Hence $\cD(A)$ contains the subalgebra of 
$\End_\Bbbk(A)$  generated by $A$, acting on itself by multiplication,  and $\Der_\Bbbk(A)$.
Some work is required to prove that these two algebras coincide in the situations that will be of interest to us.


\begin{thm} 
\label{thm.DX}
Let $X$ be a non-singular irreducible affine variety over an algebraically closed field $\Bbbk$ of characteristic zero, and
let $A$ be a localization of $\cO(X)$. Then $\cD(A)$ is generated by $A$ and $\Der_\Bbbk(A)$.
\end{thm} 

Theorem \ref{thm.DX} is due to Grothendieck \cite{egaIV-vol32}. There is also a proof in Sweedler's \cite{sweedler74}. 

\begin{thm} 
\label{thm.T*X}
Let $X$ be a non-singular irreducible affine variety over an algebraically closed field $\Bbbk$ of characteristic zero.
Let $A=\cO(X)$.
The subspaces $\cD^0(A) \subseteq  \cD^1(A) \subseteq \cdots$ provide a filtration on $\cD(X)$
and ${\sf gr}\,\cD(X) \cong \cO(T^*X)$, the coordinate ring of the cotangent bundle. 
\end{thm} 

 

\begin{cor} 
\label{cor.DX}
Let $\Bbbk$ be an algebraically closed field of characteristic zero. If $X$ is non-singular affine variety over $\Bbbk$, then
$\cD(X)$ is a left and right noetherian domain.
\end{cor} 


Theorem  \ref{thm.DX} fails when $X$ is singular. 
The first example showing this is due to Bernstein-Gel'fand-Gel'fand \cite{bgg72}. They show that 
$\cD(X)$ is not noetherian when $X$ is the surface $x^3+y^3+z^3=0$. 
However, Theorem  \ref{thm.DX} doesn't fail for all singular $X$: if $X$ is an affine curve, then $\cD(X)$ is finitely generated and noetherian \cite{ss88}.

Exercise: compute $\cD(A)$ when $A=\Bbbk[x]/(x^n)$. 



\subsection{The sheaf of differential operators $\calD_X$}



\section{(Twisted)  differential operators on $\P 1$ and representations of $\sl(2,\bbC)$}



\printbibliography

\end{document}